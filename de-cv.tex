\documentclass[10pt,a4paper]{article}

\usepackage[top=3.2cm,bottom=3.2cm,left=3.2cm,right=3.2cm]{geometry}

\usepackage{color}
	\definecolor{heading-rule}{rgb}{0.7, 0.7, 0.7}
	\definecolor{header-blue}{RGB}{51, 102, 255}

\usepackage{marginnote}
\newcommand{\timespan}[1]{\marginnote{\scriptsize #1}}
\reversemarginpar

\usepackage{url}

\usepackage{hyperref} % should be last usepackage to avoid errors.

% don't indent paragraphs, and columns in tabulars
\setlength\parindent{0em}
\setlength\tabcolsep{0em}

% make lists without bullets and compact spacing
\renewenvironment{itemize}{
	\begin{list}{}{
		\setlength{\leftmargin}{0em}
		\setlength{\itemsep}{0.25em}
		\setlength{\parskip}{0pt}
		\setlength{\parsep}{0.25em}}}
	{\end{list}}


\newcommand{\customhrule}{\vspace{0.2cm}\textcolor{heading-rule}\hrule\vspace{0.2cm}}

\usepackage{relsize}

\protected\def\cpp{{C\nolinebreak[4]\hspace{-.05em}\raisebox{.4ex}{\relsize{-3}\bfseries ++}}}

\def\name{Ronny Brendel}



\usepackage{ngerman}

% pdf meta data
\hypersetup{
%	colorlinks = true,
%	urlcolor = black,
	pdfauthor = {\name},
	pdftitle = {\name: Lebenslauf (zuletzt aktualisiert \today)},
	pdfsubject = {Lebenslauf},
	pdfpagemode = UseNone % turn off bookmarks panel in pdf readers
}

\begin{document}
\pagestyle{empty}

%%%%%%%%%%%%%%%%%%%%%%%%%%%%%%%%%%%%%%%%%%%%%%%%%%%%%%%%%%%%%%%%%%%%%%%%%%%%%%%
\color{header-blue}

	\section*{\Huge \name}

	R\"acknitzh\"ohe 32 \hfill +49 1577 51 78 737 (mobil)\\
	01217 Dresden \hfill ronnybrendel@gmail.com\\
	Deutschland \hfill

\color{black}

\vspace{0.6cm}


%%%%%%%%%%%%%%%%%%%%%%%%%%%%%%%%%%%%%%%%%%%%%%%%%%%%%%%%%%%%%%%%%%%%%%%%%%%%%%%
\section*{Pers\"onliche Informationen}
\begin{itemize}
	\item Geboren am 11.10.1985 in Mei\ss en, Deutschland
	\item Deutscher Staatsb\"urger
	\item Ledig
\end{itemize}

%%%%%%%%%%%%%%%%%%%%%%%%%%%%%%%%%%%%%%%%%%%%%%%%%%%%%%%%%%%%%%%%%%%%%%%%%%%%%%%
\customhrule
\section*{Berufserfahrung}
\begin{itemize}
	\item \timespan{2015\textendash heute}
		\textbf{TU Dresden}, Zentrum f\"ur Informationsdienste und Hochleistungsrechnen \\
		\textbf{Wissenschaftlicher Mitarbeiter}
		\item Fortschritte im Gebiet Software-Performance-Analyse f\"ur hoch-parallele Anwendungen. Ziele:
			\item \hspace{1em} Ersetzen von Post-mortem-Analyse durch Online-Analyse
			\item \hspace{1em} Fortschritte in der vergleichenden Analyse
		\item Autor der Ver\"offentlichung \emph{Structural Clustering: A New Approach to Support Performance Analysis at Scale}
	\item \timespan{2012\textendash 2013}
		\textbf{TU Dresden}, Lehrstuhl f\"ur algebraische und logische Grundlagen der Informatik \\
		\textbf{Studentische Hilfskraft}
		\begin{itemize}
			\item Betreuung zweier \"Ubungen: \textit{Theoretische Informatik und Logik}, und \textit{Advanced Logic}
			\item Entwicklung neuartiger Synchronisationsalgorithmen
		\end{itemize}
	\item \timespan{2007\textendash 2012}
		\textbf{TU Dresden}, Zentrum f\"ur Informationsdienste und Hochleistungsrechnen \\
		\textbf{Informatiker}
		\begin{itemize}
			\item Entwicklung einer Architektur zur Analyse hochparalleler Programme -- \emph{Vampir} -- \href{http://www.vampir.eu}{vampir.eu}
			\item Assistenz beim Organisieren und Koordinieren des Entwicklungsprozesses
			\item Beitr\"age zu \emph{Open Trace Format}, \emph{VampirTrace}, und mehreren kleineren Projekten
			\item Mitautor der Ver\"offentlichungen \emph{Introducing the Open Trace Format (OTF)}, 
			\emph{Memory Allocation Tracing with VampirTrace}, und \emph{Trace File Comparison with a Hierarchical Sequence Alignment Algorithm}
		\end{itemize}
\end{itemize}

%%%%%%%%%%%%%%%%%%%%%%%%%%%%%%%%%%%%%%%%%%%%%%%%%%%%%%%%%%%%%%%%%%%%%%%%%%%%%%%
\vspace{-0.2cm}
\customhrule
\section*{Bildungsweg}
\begin{itemize}
	\item \timespan{2015}
		Diplom-Informatiker, TU Dresden
	\item \hspace{1em} Nebenfach: Diskrete Mathematik, Algebra \& Geometrie
	\item \timespan{2013}
		Auslandssemester, TU Wien
	\item \timespan{2007}
		Fachinformatiker f\"ur Anwendungsentwicklung (Ausbildung), TU Dresden
	\item \timespan{2004}
		Abitur, Franziskaneum Gymnasium, Mei\ss en
\end{itemize}

%%%%%%%%%%%%%%%%%%%%%%%%%%%%%%%%%%%%%%%%%%%%%%%%%%%%%%%%%%%%%%%%%%%%%%%%%%%%%%%
\pagebreak
\section*{Technische F\"ahigkeiten}
\vspace{0.1cm} % balance out vspacing difference between tabular and itemize
\begin{tabular}{l l}
	\begin{minipage}{0.18\textwidth}
		\begin{itemize}
			\item Sprachen:
			\item Bibliotheken:
			\item Tools:
			\item Sonstiges:
		\end{itemize}
	\end{minipage}
	&
	\begin{minipage}{0.80\textwidth}
		\begin{itemize}
			\item C, \cpp, \cpp11/14, Go, Python, Java, Haskell, Lisp, und viele mehr
			\item Qt, STL, OpenMP, Message Passing Interface, Django
			\item Vim, Bash, Git, zahlreiche \textasteriskcentered{}nix-Tools, Valgrind, \LaTeX, TikZ, Visual Studio
			\item Hochleistungsrechnen, Formale Methoden, Algorithmen, API-, UI design
		\end{itemize}
	\end{minipage}
\end{tabular}

\vspace{0.66cm} % balance out vspacing difference between tabular and itemize

%%%%%%%%%%%%%%%%%%%%%%%%%%%%%%%%%%%%%%%%%%%%%%%%%%%%%%%%%%%%%%%%%%%%%%%%%%%%%%%
\customhrule
\section*{Interessen \& Aktivit\"aten}
\begin{itemize}
	\item \cpp11/14, diskrete Mathematik, Software-Performance-Analyse, besser Software entwickeln
	\item Private Programmierprojekte: \href{https://github.com/hydroo}{github.com/hydroo}
	\item Ich habe 136 mathematische Programmierprobleme gel\"ost: \href{http://projecteuler.net/profile/hydro.png}{projecteuler.net/profile/hydro.png}
	\item Erholung: Schwimmen, Laufen, Lesen, Computerspiele, IT- und Wirtschaftsnachrichten
\end{itemize}

%%%%%%%%%%%%%%%%%%%%%%%%%%%%%%%%%%%%%%%%%%%%%%%%%%%%%%%%%%%%%%%%%%%%%%%%%%%%%%%
\customhrule
\section*{Sprachen}
\begin{itemize}
	\item Deutsch (Muttersprache), Englisch (flie\ss end)
\end{itemize}

%%%%%%%%%%%%%%%%%%%%%%%%%%%%%%%%%%%%%%%%%%%%%%%%%%%%%%%%%%%%%%%%%%%%%%%%%%%%%%%
\end{document}
